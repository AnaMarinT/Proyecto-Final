\documentclass{article}
\usepackage[utf8]{inputenc}
\usepackage[spanish]{babel}
\usepackage{listings}
\usepackage{graphicx}
\graphicspath{ {images/} }
\usepackage{cite}

\begin{document}

\begin{titlepage}
    \begin{center}
        \vspace*{1cm}
            
        \Huge
        \textbf{Proyecto Final}
            
        \vspace{0.5cm}
        \LARGE
        Los primeros pasos
            
        \vspace{1.5cm}
            
        \textbf{Ana María Marín Toro}
            
        \vfill
            
        \vspace{0.8cm}
            
        \Large
        Despartamento de Ingeniería Electrónica y Telecomunicaciones\\
        Universidad de Antioquia\\
        Medellín\\
        Marzo de 2021
            
    \end{center}
\end{titlepage}

\newpage

\section{Introducción}
La siguiente actividad constituye el paso inicial del desarrollo del proyecto final del curso. Dicha idea estará sujeta a cambios conforme se avance en los temas del curso, se exploren otras opciones y se conozcan las limitaciones del proyecto.

\section{Idea Inicial} 
La opción que deseo implementar como proyecto final es un videojuego de plataformas. Se llamará “El Isleño” y estará inspirado en la saga  Adventure Island de la compañía Hudson Soft. Dentro de las modificaciones que incluirá mi versión, se encuentran: la opción multijugador, una menor cantidad de niveles, cambios en la historia y la combinación de diferentes elementos de cada uno de los juegos de la saga en uno solo, esto brindara un poco de libertad, además de ofrecer una mayor variedad de opciones a la hora de implementar los modelos físicos requeridos por el proyecto.\\\

Existen dos razones principales que me llevan a decidir no desarrollar un juego original, la primera es que Adventure Island es uno de mis juegos favoritos y la idea de poder personalizarlo me llama mucho la atención, y la segunda – y probablemente la más importante – la mayor parte de lo relacionado a la parte de gráficos ya se encuentra desarrollado y disponible en línea (respetando derechos de autor), lo que va a reducir la carga del proyecto – considerando que trabajare sola – y me permitirá enfocarme en aspectos más relacionados con el código del proyecto.


\end{document}
